\documentclass[a4paper, 12pt]{article}
\usepackage[utf8]{inputenc}
\usepackage[T2A]{fontenc}
\usepackage{amsmath,amssymb,amsthm}
\usepackage[a4paper,hmargin=2.5cm,vmargin=2.5cm]{geometry}
\usepackage[english,russian]{babel}

\begin{document}
%%
%% Title page
%%
\begin{center}
{\scshape Федеральное государственное автономное\\
образовательное учреждение высшего образования\\
<<Национальный исследовательский университет\\
<<Высшая школа экономики>>\\[1ex]
Факультет математики\par}

\par\vfill

\textbf{\large Гузеев Виталий Вячеславович}

\vspace{1.5cm}

{\Large\bfseries
Соотношения Плюккера
\par}

\vspace{1.5cm}

Курсовая работа студента 1 курса\\[1ex]
образовательной программы бакалавриата <<Математика>>
\par\vfill
\noindent\hspace{0.52\textwidth}\parbox[t]{0.48\textwidth}{%
Научный руководитель:\\[3pt]
кандидат физико-математических наук,\\
профессор\\
Городенцев Алексей Львович\\[2ex]
}%
\par\vfill
Москва 2019
\end{center}
\thispagestyle{empty}
\pagebreak
%%
%% ===========================================================================
%%
\section{Теорема Безу}
\subsection{Базовые определения и формулировки}
\begin{description}
\item[Проективная плоскость]
Рассмотрим пространство $\mathbb{C}^3$. Прямые $\mathbb{C}^3$, проходящие через 0 $\mathbb{C}^3$, образуют двумерное пространство, называемое комплексной (далее подразумевается) проективной плоскостью.
\item[Плоская (алгебраическая) кривая]
Множество точек плоскости, в частности, проективной, являющихся нулями некоторого многочлена, будем называть плоской кривой.
% Однородный многочлен - определение через линейность, почему иные определения не имеют смысла.
% Точка пересечения кривых
% Формулировка теоремы
\end{description}
\subsection{Доказательство}
% результант
% смысл его
% Почему работает для нахождения общих корней.
% Ну и сам он степени mn
\subsection{Правило Цойтена}
% Описание рисунка
% Некий аналитический этюд
\section{Соотношения Плюккера}
\subsection{Базовые определения и формулировки}
% Определения всех действующих лиц. Начиная с двойственной кривой.
\section{Дальнейшие обобщения}
\end{document}
