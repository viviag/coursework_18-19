\documentclass[a4paper, 12pt]{article}
\usepackage[utf8]{inputenc}
\usepackage[T2A]{fontenc}
\usepackage{amsmath,amssymb,amsthm}
\usepackage[unicode]{hyperref}
\usepackage[a4paper,hmargin=2.5cm,vmargin=2.5cm]{geometry}
\usepackage[english,russian]{babel}
\overfullrule=2cm

\begin{document}
%%
%% Title page
%%
\begin{center}
{\scshape Федеральное государственное автономное\\
образовательное учреждение высшего образования\\
<<Национальный исследовательский университет\\
<<Высшая школа экономики>>\\[1ex]
Факультет математики\par}

\par\vfill

\textbf{\large Гузеев Виталий Вячеславович}

\vspace{1.5cm}

{\Large\bfseries
Соотношения Плюккера
\par}

\vspace{1.5cm}

Курсовая работа студента 1 курса\\[1ex]
образовательной программы бакалавриата <<Математика>>
\par\vfill
\noindent\hspace{0.52\textwidth}\parbox[t]{0.48\textwidth}{%
Научный руководитель:\\[3pt]
кандидат физико-математических наук,\\
профессор\\
Городенцев Алексей Львович\\[2ex]
}%
\par\vfill
Москва 2019
\end{center}
\thispagestyle{empty}
\pagebreak
%%
%% ===========================================================================
%%
\section{Теорема Безу}
\subsection{Базовые определения и формулировки}
\begin{description}
\item[Проективная плоскость]
Рассмотрим векторное пространство $\mathbb{C}^3$. Ненулевые векторы $\mathbb{C}^3$, рассматриваемые с точностью до пропорциональности, образуют двумерное пространство, называемое комплексной (далее подразумевается) проективной плоскостью и обозначаемое $\mathbb{C}P^2$.
\item[Глобальные однородные координаты]
Координаты коллинеарных векторов в\newline произвольном базисе связаны соотношениями $x_i = \lambda y_i$ $\forall i \in {0..2}$. То есть можно говорить о задании точки проективной плоскости отношением координат векторов $\mathbb{C}$ $(x_0:x_1:x_2)$.
\end{description}
\begin{description}
\item[Однородные многочлены] Однородными называются многочлены, все мономы\newline которых имеют одинаковую степень. Множество нулей однородного многочлена задаёт некоторое множество точек на проективной плоскости. Свойство однородных многочленов --- $f(x)$ = $0$ => $f(\lambda x)$ = $0$.
\item[Плоская (алгебраическая) кривая]
Множество точек проективной плоскости,\newline являющихся нулями некоторого однородного многочлена, будем называть\newline плоской кривой. Степенью кривой будем называть степень задающего её многочлена. Будем говорить также, что многочлен $f$ задаёт кривую $F$, если кривая $F$ представляет собой множество нулей $f$.
\item[Точка пересечения кривых]
Точкой пересечения $n$ плоских кривых будем называть общий ноль $n$ многочленов, задающих означенные кривые.
\end{description}
Введённых определений достаточно, чтобы сформулировать для комплексной проективной плоскости \textbf{слабую теорему Безу}:
\textit{Две плоские кривые степеней $m$ и $n$, имеющие конечное количество точек пересечения, пересекаются не более чем в $mn$ точках}
\subsection{Доказательство слабой теоремы Безу}
% Хочу уметь на этом месте строить аппарат под доказательство через распад кривой степени n на n прямых. Нашёл у Арнольда здесь (https://books.google.cz/books?id=wrDDBgAAQBAJ&pg=PT87&lpg=PT87&dq=%D1%82%D0%B5%D0%BE%D1%80%D0%B5%D0%BC%D0%B0+%D0%B1%D0%B5%D0%B7%D1%83+%D1%82%D0%BE%D0%BF%D0%BE%D0%BB%D0%BE%D0%B3%D0%B8%D1%87%D0%B5%D1%81%D0%BA%D0%BE%D0%B5+%D0%B4%D0%BE%D0%BA%D0%B0%D0%B7%D0%B0%D1%82%D0%B5%D0%BB%D1%8C%D1%81%D1%82%D0%B2%D0%BE&source=bl&ots=Co0eCf3004&sig=v6v_5eIJMNjOoiE9qcaSPrB4om8&hl=en&sa=X&ved=2ahUKEwikqs6TlcjfAhXCI1AKHet7BawQ6AEwCHoECAMQAQ#v=onepage&q=%D1%82%D0%B5%D0%BE%D1%80%D0%B5%D0%BC%D0%B0%20%D0%B1%D0%B5%D0%B7%D1%83%20%D1%82%D0%BE%D0%BF%D0%BE%D0%BB%D0%BE%D0%B3%D0%B8%D1%87%D0%B5%D1%81%D0%BA%D0%BE%D0%B5%20%D0%B4%D0%BE%D0%BA%D0%B0%D0%B7%D0%B0%D1%82%D0%B5%D0%BB%D1%8C%D1%81%D1%82%D0%B2%D0%BE&f=false).
% Верно ли, что отказ от геометрических построений вовсе приведёт меня к дивизорам на схемах?

\textbf{Теорема}
\smallskip
Обозначим $Syl(f,g)$ (матрица Сильвестра (\href{https://ru.wikipedia.org/wiki/%D0%A1%D0%B8%D0%BB%D1%8C%D0%B2%D0%B5%D1%81%D1%82%D1%80,_%D0%94%D0%B6%D0%B5%D0%B9%D0%BC%D1%81_%D0%94%D0%B6%D0%BE%D0%B7%D0%B5%D1%84}{Джозеф Сильвестр})) =

Рассмотрим многочлены $f$, $g$ от одной переменной с коэффициентами $f_i$, $g_i$ в $\mathbb{C}$.
\smallskip
$\begin{pmatrix}
g_m & g_{m-1} & g_{m-2} & \cdots & \cdots & g_0 & 0 & \cdots & 0 \\
0 & g_m & g_{m-1} & g_{m-2} & \cdots & g_1 & g_0 & \cdots & 0 \\
\cdots & \cdots & \cdots & \cdots & \cdots & \cdots & \cdots & \cdots & \cdots \\
0 & \cdots & 0 & g_m & g_{m-1} & g_{m-2} & \cdots & g_1 & g_0 \\
f_n & f_{n-1} & \cdots & \cdots & f_0 & 0 & \cdots & \cdots & 0 \\
0 & f_n & \cdots & \cdots & f_1 & f_0 & 0 & \cdots & 0 \\
\cdots & \cdots & \cdots & \cdots & \cdots & \cdots & \cdots & \cdots & \cdots \\
0 & \cdots & \cdots & \cdots & 0 & f_n & \cdots & f_1 & f_0
\end{pmatrix}$
% Матрица с Википедии.
\smallskip

Тогда у $f$ и $g$ существует общий корень тогда и только тогда, когда результант этих многочленов = $Res(f,g) := det(Syl(f,g)) = 0$.
\bigskip

\textbf{Лемма}

\textit{ Многочлены от одной переменной $f$ и $g$ имеют общий корень над $\mathbb{C}$ тогда и только тогда, когда $\exists \phi(x)$, $\psi(x)$, $deg(\phi) < m$, $deg(\psi) < n$: $\phi(x)f = \psi(x)g$}

\smallskip
\textbf{Доказательство леммы:}
\smallskip
(=>): Для многочленов от одной переменной над полем имеет место теорема Безу о делимости многочлена на одночлен $(x - x_0)$, где $x_0$ --- корень. То есть из существования общего корня $x_0$ у многочленов $f$ и $g$ следует, что $\exists h(x)=(x-x_0)$, $\phi(x)$, $\psi(x)$, $deg(\phi) = m-1$, $deg(\psi) = n-1$: $g(x)=h(x)\phi(x)$, $f(x)=h(x)\psi(x)$ => $g(x)\psi(x) = f(x)\phi(x)$ $\qed$
\smallskip
(<=): В разложение $\Phi(x) = \phi(x)f(x) = \psi(x)g(x)$ на неприводимые входят все неприводимые множители $f$ и $g$. Но $deg(\Phi) < m + n$, то есть среди этих множителей есть хотя бы один общий. Следовательно, его корень - общий для $f$ и $g$. $\qed$

\textbf{Доказательство теоремы:}
\smallskip
Записав полученное выражение, получаем равенство $(f_nx^n + ... + f_0)\cdot(\phi_{m-1}x^{m-1} + ...) = (g_mx^m + ... + g_0)\cdot(\psi_{n-1}x^{n-1} + ...)$.\newline
Расписав выражение по степеням $x$, получим систему уравнений на коэффиценты $\phi$, $\psi$, матрица которой совпадает с матрицей Сильвестра.\newline
Полученное утверждение совпадает с теоремой. \qed
\bigskip

Обозначения многочленов из теоремы удобны и более для многочленов от одной переменной использованы не будут. Так что рассмотрим теперь однородные многочлены $f$ и $g$ от трёх переменных $x,y,z$, задающие некоторые плоские кривые на $\mathbb{C}P^2$.\newline
Будем рассматривать эти многочлены как многочлены от $x$ с коэффициентами --- мономами от $y,z$. Обозначим их для удобства $f'$ и $g'$.\newline
Обозначим функцию от переменных $y,z$, значением которой является\newline $Res(f'(y,z),g'(y,z))$ как $Res(f',g')$ = $det(Syl(f',g'))$

\smallskip
\textbf{Теорема}

\textit{$Res(f',g')$ многочленов степеней $m$ и $n$ соответственно --- однородный многочлен степени $mn$ от $(y,z)$}

\textbf{Доказательство теоремы:}

$Syl(f',g')(\lambda y,\lambda z)$ =

$\begin{pmatrix}
g_m & \lambda g_{m-1} & \lambda^{2} g_{m-2} & \cdots & \cdots & \lambda^{m}g_0 & 0 & \cdots & 0 \\
0 & g_m & \lambda g_{m-1} & \lambda^{2} g_{m-2} & \cdots & \lambda^{m-1} g_1 & \lambda^{m} g_0 & \cdots & 0 \\
\cdots & \cdots & \cdots & \cdots & \cdots & \cdots & \cdots & \cdots & \cdots \\
0 & \cdots & 0 & g_m & \lambda g_{m-1} & \lambda^{2} g_{m-2} & \cdots & \lambda^{m-1} g_1 & \lambda^{m} g_0 \\
f_n & \lambda*f_{n-1} & \cdots & \cdots & \lambda^{n}*f_0 & 0 & \cdots & \cdots & 0 \\
0 & f_n & \cdots & \cdots & \lambda^{n-1} f_1 & \lambda^{n} f_0 & 0 & \cdots & 0 \\
\cdots & \cdots & \cdots & \cdots & \cdots & \cdots & \cdots & \cdots & \cdots \\
0 & \cdots & \cdots & \cdots & 0 & f_n & \cdots & \lambda^{n-1} f_1 & \lambda^{n} f_0
\end{pmatrix}$\newline
$\forall$ $y,z$: $Syl(f',g')(y,z)$ = $0$. Это следует из свойства однородных многочленов для многочленов от $x,y,z$, полученных из $f'$ и $g'$ домножением на дополнительную переменную $x$ в соответствующей степени.
\smallskip
Домножим первые $n$ строк на 1, $\lambda$, $\dots$, $\lambda^{n-1}$ соответственно, оставшиеся $m$ строк на 1, $\lambda$, $\dots$, $\lambda^{m-1}$ (учтя соответствующие изменения определителя, который поделится на $\lambda^{1 +\dots + (n-1)}$ и на $\lambda^{1 +\dots + (m-1)}$).
Тем самым столбцы матрицы домножились на 1, $\lambda$, $\dots$, $\lambda^{m + n - 1}$ соответственно.\newline
В итоге $Res(f',g')(\lambda y,\lambda z)$ = $\lambda^{p-q-r}Res(F,G)(y,z)$, где $p = (m+n)(m+n-1)/2$, $q = m(m-1)/2 $, $r = n(n-1)/2$ = $mn$.
\qed

\bigskip
Сопоставим $Res$ неоднородный многочлен $R$ от одной переменной $z$ той же степени, положив, например, $y=1$. Возможно, для этого необходимо линейно перейти к другим переменным, результант от этого не изменится.\newline
$R(z)$ - многочлен от одной переменной с комплексными коэффициентами, то есть к нему применима основная теорема алгебры и он имеет $mn$ корней.\newline
$Res(y,z)$ - однородный многочлен, то есть любому его корню $(y_0, z_0)$ соотвествует корень $(1, z_0/y_0)$ $(y_0 \neq 0$ при $z \neq 0$ --- отдельное требование к замене переменных результанта.)
По построению $R(z_0/y_0) = 0$.\newline
Пусть теперь $\exists x: R(x) = 0$. Тогда точка $(1:x)$ --- корень $Res(y,z)$ по тому же построению.
То есть $Res(y,z)$ имеет ровно $mn$ корней.

\bigskip
\textbf{Теорема}
\textit{Две плоские кривые степеней $m$ и $n$, имеющие конечное количество точек пересечения, пересекаются не более чем в $mn$ точках}\newline
\smallskip
\textbf{Доказательство теоремы:}\newline
Рассмотрим многочлены $f$ и $g$, задающие кривые $F(x,y,z) = 0$ и $G(x,y,z) = 0$. Если результант этих многочленов тождественно равен нулю, многочлены имеют бесконечно много общих корней, следовательно, кривые $F$ и $G$ имеют бесконечно много точек пересечения. Иначе результант имеет $mn$ корней, следовательно, различных пар координат $(y,z)$ общих корней многочленов не более $mn$ кривые пересекаются в не более чем $mn$ точках, что следует из процедуры построения результанта для многочленов от одной переменной.

\bigskip
Далее пока план:\newline
2. Теорема о разложении результанта на линейные множители.\newline
3. Начало следующей секции - сильная теорема Безу. \newline
4. Кратность пересечения, аналитическое определение.\newline
5. Теорема о том, что это число целое.\newline
6. Теорема о совпадении этого числа с алгебраической кратностью линейного множителя результанта.\newline
7. Сильная теорема Безу\newline
8. Дальше что-то про Плюккера.\newline

\end{document}
