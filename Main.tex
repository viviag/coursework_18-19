\documentclass[a4paper, 12pt]{article}
\usepackage[utf8]{inputenc}
\usepackage[T2A]{fontenc}
\usepackage{amsmath,amssymb,amsthm}
\usepackage{hyperref}
\usepackage[a4paper,hmargin=2.5cm,vmargin=2.5cm]{geometry}
\usepackage[english,russian]{babel}

\begin{document}
%%
%% Title page
%%
\begin{center}
{\scshape Федеральное государственное автономное\\
образовательное учреждение высшего образования\\
<<Национальный исследовательский университет\\
<<Высшая школа экономики>>\\[1ex]
Факультет математики\par}

\par\vfill

\textbf{\large Гузеев Виталий Вячеславович}

\vspace{1.5cm}

{\Large\bfseries
Соотношения Плюккера
\par}

\vspace{1.5cm}

Курсовая работа студента 1 курса\\[1ex]
образовательной программы бакалавриата <<Математика>>
\par\vfill
\noindent\hspace{0.52\textwidth}\parbox[t]{0.48\textwidth}{%
Научный руководитель:\\[3pt]
кандидат физико-математических наук,\\
профессор\\
Городенцев Алексей Львович\\[2ex]
}%
\par\vfill
Москва 2019
\end{center}
\thispagestyle{empty}
\pagebreak
%%
%% ===========================================================================
%%
\section{Теорема Безу}
% Считаю следующую подсекцию примерно завершённой.
\subsection{Базовые определения и формулировки}
\begin{description}
\item[Проективная плоскость]
Рассмотрим векторное пространство $\mathbb{C}^3$. Ненулевые векторы $\mathbb{C}^3$, рассматриваемые с точностью до пропорциональности, образуют двумерное пространство, называемое комплексной (далее подразумевается) проективной плоскостью.
\item[Глобальные однородные координаты]
Координаты коллинеарных векторов в произвольном базисе связаны соотношениями $x_i=\lambda*y_i  \forall i \in {0..2}$. То есть можно говорить о задании точки проективной плоскости соотношением координат векторов порождающего её пространства $(x_0:x_1:x_2)$.
Произвольный многочлен f от трёх переменных над $\mathbb{C}$ не обязан задавать множество точек на проективной плоскости, так как для точки x: f(x) = 0 не гарантируется, что f($\lambda$x) = 0, то есть множество решений может не задавать никаких точек проективной плоскости.
\item[Однородные многочлены] Назовём однородными многочлены, для которых f(x) = 0 => f($\lambda$x) = 0. Альтернативное определение - однородными называются многочлены, все мономы которых имеют одинаковую степень. Эквивалентность определений проверяется напрямую - $\lambda$ в каждый моном войдёт в степени, равной степени монома. Множество нулей однородного многочлена задаёт некоторое множество точек на проективной плоскости.
\item[Плоская (алгебраическая) кривая]
Множество точек плоскости, в частности, проективной, являющихся нулями некоторого многочлена, будем называть плоской кривой. Степенью кривой будем называть степень задающего её многочлена.
\item[Точка пересечения кривых]
Точкой пересечения n плоских кривых будем называть общий ноль n многочленов, задающих означенные кривые. Кратность пересечения - минимальная из кратностей общего корня этих многочленов.
\end{description}
Введённых определений достаточно, чтобы сформулировать для комплексной проективной плоскости \textbf{теорему Безу}:
\textit{Две плоские кривые степеней m и n, имеющие конечное количество точек пересечения, пересекаются с учётом кратности пересечений ровно в mn точках}
Заметим, что вся построенная терминология прямо переносится на n-мерный случай с заменой понятия "проективная плоскость" на понятие "проективное пространство".
\subsection{Доказательство}
% Хочу уметь на этом месте строить аппарат под доказательство через распад кривой степени n на n прямых. Нашёл у Арнольда здесь (https://books.google.cz/books?id=wrDDBgAAQBAJ&pg=PT87&lpg=PT87&dq=%D1%82%D0%B5%D0%BE%D1%80%D0%B5%D0%BC%D0%B0+%D0%B1%D0%B5%D0%B7%D1%83+%D1%82%D0%BE%D0%BF%D0%BE%D0%BB%D0%BE%D0%B3%D0%B8%D1%87%D0%B5%D1%81%D0%BA%D0%BE%D0%B5+%D0%B4%D0%BE%D0%BA%D0%B0%D0%B7%D0%B0%D1%82%D0%B5%D0%BB%D1%8C%D1%81%D1%82%D0%B2%D0%BE&source=bl&ots=Co0eCf3004&sig=v6v_5eIJMNjOoiE9qcaSPrB4om8&hl=en&sa=X&ved=2ahUKEwikqs6TlcjfAhXCI1AKHet7BawQ6AEwCHoECAMQAQ#v=onepage&q=%D1%82%D0%B5%D0%BE%D1%80%D0%B5%D0%BC%D0%B0%20%D0%B1%D0%B5%D0%B7%D1%83%20%D1%82%D0%BE%D0%BF%D0%BE%D0%BB%D0%BE%D0%B3%D0%B8%D1%87%D0%B5%D1%81%D0%BA%D0%BE%D0%B5%20%D0%B4%D0%BE%D0%BA%D0%B0%D0%B7%D0%B0%D1%82%D0%B5%D0%BB%D1%8C%D1%81%D1%82%D0%B2%D0%BE&f=false).
Пусть f и g - однородные многочлены с комплексными коэффициентами степеней n и m от трёх переменных, задающие на $\mathbb{C}P^2$ две различные плоские кривые F и G.
Запишем их как многочлены от одной переменной $f=f_n*x^n + ... + f_0, g=g_m*x^m + ... + g_0, f_i \in \mathbb{C}[y,z] соответствующей степени$.

\textbf{Лемма 1}

\textit{f и g имеют общий корень над $\mathbb{C}$ тогда и только тогда, когда $\exists \phi(x), \psi(x), deg(\phi) < m, deg(\psi) < n: \phi(x)*f = \psi(x)*g$}

\textbf{Доказательство леммы 1:}

(=>): Для многочленов от одной переменной над полем имеет место соответствующая теорема Безу (обобщением которой в некотором смысле является рассматриваемая). То есть из существования общего корня $x_0$ у многочленов f и g следует, что $\exists h(x) (зависящий от x_0), \phi(x), \psi(x), deg(\phi) < m, deg(\psi) < n: g(x)=h(x)\phi(x), f(x)=h(x)\psi(x) => g(x)\psi(x) = f(x)\phi(x) \qed$

(<=): В разложение $\Phi(x) = \phi(x)f(x) = \psi(x)g(x)$ на неприводимые входят все неприводимые множители f и g. Но $deg(\Phi) < m + n$, то есть среди этих множителей есть хотя бы один общий. Следовательно, его корень - общий для f и g. $\qed$

Рассуждение применимо к кривым F и G, если f и g $\forall фиксированных y_0 и z_0$ имеют те же степени, что и кривые. То есть $моном kx^n входит в f, lx^m в g и f_n*g_m != 0$. Применительно к кривым это означает, что ни одна из них не содержит точку (1:0:0), при этом это условие не зависит от y и z. Очевидно, для любой кривой существует система координат, в которой она не содержит единственную заданную координатами точку.

Записав полученное выражение, получаем равенство $(f_n*x^n + ... + f_0)*(\phi_m-1*x^(m-1) + ...) = (g_m*x^m + ... + g_0)(\psi_(n-1)*x^(n-1) + ...) для произвольных фиксированных y_0, z_0$
Расписав выражение по степеням x, получим условие наличия у двух кривых точки пересечения:

Обозначим Syl(F,G)(x,y) (матрица Сильвестра (\href{https://ru.wikipedia.org/wiki/%D0%A1%D0%B8%D0%BB%D1%8C%D0%B2%D0%B5%D1%81%D1%82%D1%80,_%D0%94%D0%B6%D0%B5%D0%B9%D0%BC%D1%81_%D0%94%D0%B6%D0%BE%D0%B7%D0%B5%D1%84}{Джозеф Сильвестр})) =

$\begin{pmatrix}
\phi_m & \phi_{m-1} & \phi_{m-2} & \cdots & \cdots & \phi_0 & 0 & \cdots & 0 \\
0 & \phi_m & \phi_{m-1} & \phi_{m-2} & \cdots & \phi_1 & \phi_0 & \cdots & 0 \\
\cdots & \cdots & \cdots & \cdots & \cdots & \cdots & \cdots & \cdots & \cdots \\
0 & \cdots & 0 & \phi_m & \phi_{m-1} & \phi_{m-2} & \cdots & \phi_1 & \phi_0 \\
\psi_n & \psi_{n-1} & \cdots & \cdots & \psi_0 & 0 & \cdots & \cdots & 0 \\
0 & \psi_n & \cdots & \cdots & \psi_1 & \psi_0 & 0 & \cdots & 0 \\
\cdots & \cdots & \cdots & \cdots & \cdots & \cdots & \cdots & \cdots & \cdots \\
0 & \cdots & \cdots & \cdots & 0 & \psi_n & \cdots & \psi_1 & \psi_0
\end{pmatrix}$
% Матрица с Википедии.

$Пусть f=f_n*x^n + ... + f_0, g=g_m*x^m + ... + g_0, f_i \in \mathbb{C}[y,z] соответствующей степени - многочлены, определяющие кривые F и G$,
Тогда $\exists (x_0:y_0:z_0)$ - точка пересечения F и G тогда и только тогда, когда $результант = Res(F,G) = det(Syl(F,G))(y_0,z_0) = 0$, то есть определяемая им система имеет ненулевое решение. Эта точка единственна для каждого нуля результанта, так как $\phi(x) и \psi(x)$ по построению зависят от $x_0$, то есть разным точкам пересечения соответствуют разные решения системы Сильвестра.

Заметим, что все рассуждения проходят в однородных координатах и обобщимы на случай произвольной размерности.

\textbf{Лемма 2}

\textit{Res(F,G) кривых степеней m и n соответственно - однородный многочлен степени mn от (y,z).}

% доказательство

\textbf{Лемма 3}

\textit{Req(F,G) $\equiv$ 0 <=> F и G имеют бесконечное число точек пересечения.}

%доказательство

\subsection{Наглядный смысл}
%%\item[Аффинная карта и локальные аффинные координаты]
%%Чтобы сохранить возможность рассуждать о проективной плоскости и кривых на ней образно, введём понятие аффинной карты. \item[Афинной картой] $\mathbb{C}P^2$ называется плоскость $\mathbb{C}^3$, не проходящая через 0 (в аффинном пространстве, связанном с $\mathbb{C}^3$). Точки проективной плоскости изображаются на ней пересечениями соответствующих прямых с ней. Тогда кривая на аффинной карте соответствует некоторой кривой проективной плоскости с точностью до плоскости, параллельной карте и проходящей через 0 - прямые этой плоскости не пересекаются с картой, такая плоскость (а точнее, прямые, проходящие через 0 на ней) называется бесконечно удалённой (проективной) прямой для карты.
%%Аффинная карта порождается пространством решений некоторого неоднородного уравнения v(x) = c. Функционалы на карте, дополняющие v до базиса двойственного к $\mathbb{C}^3$ пространства задают локальные аффинные координаты.

% Блок про аффинные карты закомментировал, надеясь, что он мне не понадобится нигде далее. Не хочу смешивать здесь и далее аффинные пространства с векторными, а рисунки, получающиеся на аффинных картах, обобщать сложнее, чем сразу думать в терминах, устойчивых к изменению размерности.
% Здесь также хорошо будет смотреться описание результанта в аффинных картах.
% Описание рисунка
% Некий аналитический этюд про правило Цойтена.
\section{Соотношения Плюккера}
\subsection{Базовые определения и формулировки}
% Определения всех действующих лиц. Начиная с двойственной кривой.
\section{Дальнейшие обобщения}
\end{document}
