\documentclass[a4paper, 12pt]{article}
\usepackage[utf8]{inputenc}
\usepackage[T2A]{fontenc}
\usepackage{amsmath,amssymb,amsthm}
\usepackage{hyperref}
\usepackage[a4paper,hmargin=2.5cm,vmargin=2.5cm]{geometry}
\usepackage[english,russian]{babel}

\begin{document}
%%
%% Title page
%%
\begin{center}
{\scshape Федеральное государственное автономное\\
образовательное учреждение высшего образования\\
<<Национальный исследовательский университет\\
<<Высшая школа экономики>>\\[1ex]
Факультет математики\par}

\par\vfill

\textbf{\large Гузеев Виталий Вячеславович}

\vspace{1.5cm}

{\Large\bfseries
Соотношения Плюккера
\par}

\vspace{1.5cm}

Курсовая работа студента 1 курса\\[1ex]
образовательной программы бакалавриата <<Математика>>
\par\vfill
\noindent\hspace{0.52\textwidth}\parbox[t]{0.48\textwidth}{%
Научный руководитель:\\[3pt]
кандидат физико-математических наук,\\
профессор\\
Городенцев Алексей Львович\\[2ex]
}%
\par\vfill
Москва 2019
\end{center}
\thispagestyle{empty}
\pagebreak
%%
%% ===========================================================================
%%
\section{Теорема Безу}
% Считаю следующую подсекцию примерно завершённой.
\subsection{Базовые определения и формулировки}
\begin{description}
\item[Проективная плоскость]
Рассмотрим векторное пространство $\mathbb{C}^3$. Ненулевые векторы $\mathbb{C}^3$, рассматриваемые с точностью до пропорциональности, образуют двумерное пространство, называемое комплексной (далее подразумевается) проективной плоскостью.
\item[Глобальные однородные координаты]
Координаты коллинеарных векторов в произвольном базисе связаны соотношениями $x_i = \lambda y_i$ $\forall i \in {0..2}$. То есть можно говорить о задании точки проективной плоскости соотношением координат векторов порождающего её пространства $(x_0:x_1:x_2)$.\newline
\end{description}
\bigskip
Произвольный многочлен $f$ от трёх переменных над $\mathbb{C}$ не обязан задавать множество точек на проективной плоскости, так как для точки $x: f(x) = 0$ не гарантируется, что $f(\lambda x)$ = 0.
\begin{description}
\item[Однородные многочлены] Назовём однородными многочлены, для которых $f(x) = 0 => f(\lambda x)$ = 0. Альтернативное определение - однородными называются многочлены, все мономы которых имеют одинаковую степень. Эквивалентность определений проверяется напрямую - $\lambda$ в каждый моном войдёт в степени, равной степени монома. Множество нулей однородного многочлена задаёт некоторое множество точек на проективной плоскости.
\item[Плоская (алгебраическая) кривая]
Множество точек плоскости, в частности, проективной, являющихся нулями некоторого многочлена, будем называть плоской кривой. Степенью кривой будем называть степень задающего её многочлена.
\item[Точка пересечения кривых]
Точкой пересечения $n$ плоских кривых будем называть общий ноль $n$ многочленов, задающих означенные кривые. Кратность пересечения - минимальная из кратностей общего корня этих многочленов.
\end{description}
Введённых определений достаточно, чтобы сформулировать для комплексной проективной плоскости \textbf{теорему Безу}:
\textit{Две плоские кривые степеней $m$ и $n$, имеющие конечное количество точек пересечения, пересекаются с учётом кратности пересечений ровно в $mn$ точках}\newline
Заметим, что вся построенная терминология прямо переносится на $n$-мерный случай с заменой понятия "проективная плоскость" на понятие "проективное пространство".
\subsection{Доказательство}
% Хочу уметь на этом месте строить аппарат под доказательство через распад кривой степени n на n прямых. Нашёл у Арнольда здесь (https://books.google.cz/books?id=wrDDBgAAQBAJ&pg=PT87&lpg=PT87&dq=%D1%82%D0%B5%D0%BE%D1%80%D0%B5%D0%BC%D0%B0+%D0%B1%D0%B5%D0%B7%D1%83+%D1%82%D0%BE%D0%BF%D0%BE%D0%BB%D0%BE%D0%B3%D0%B8%D1%87%D0%B5%D1%81%D0%BA%D0%BE%D0%B5+%D0%B4%D0%BE%D0%BA%D0%B0%D0%B7%D0%B0%D1%82%D0%B5%D0%BB%D1%8C%D1%81%D1%82%D0%B2%D0%BE&source=bl&ots=Co0eCf3004&sig=v6v_5eIJMNjOoiE9qcaSPrB4om8&hl=en&sa=X&ved=2ahUKEwikqs6TlcjfAhXCI1AKHet7BawQ6AEwCHoECAMQAQ#v=onepage&q=%D1%82%D0%B5%D0%BE%D1%80%D0%B5%D0%BC%D0%B0%20%D0%B1%D0%B5%D0%B7%D1%83%20%D1%82%D0%BE%D0%BF%D0%BE%D0%BB%D0%BE%D0%B3%D0%B8%D1%87%D0%B5%D1%81%D0%BA%D0%BE%D0%B5%20%D0%B4%D0%BE%D0%BA%D0%B0%D0%B7%D0%B0%D1%82%D0%B5%D0%BB%D1%8C%D1%81%D1%82%D0%B2%D0%BE&f=false).
% Верно ли, что отказ от геометрических построений вовсе приведёт меня к дивизорам на схемах?
Пусть $f$ и $g$ - однородные многочлены с комплексными коэффициентами степеней $n$ и $m$ от трёх переменных, задающие на $\mathbb{C}P^2$ две различные плоские кривые $F$ и $G$.
Запишем их как многочлены от одной переменной $f = f_n*x^n + ... + f_0$, $g = g_m*x^m + ... + g_0$, $f_i \in \mathbb{C}[y,z]$, $g_i \in \mathbb{C}[y,z]$ соответствующей степени.

\bigskip
\textbf{Лемма 1}

\textit{$f$ и $g$ имеют общий корень над $\mathbb{C}$ тогда и только тогда, когда $\exists \phi(x)$, $\psi(x)$, $deg(\phi) < m$, $deg(\psi) < n$: $\phi(x)f = \psi(x)g$}

\smallskip
\textbf{Доказательство леммы 1:}
\smallskip
(=>): Для многочленов от одной переменной над полем имеет место соответствующая теорема Безу (обобщением которой в некотором смысле является рассматриваемая). То есть из существования общего корня $x_0$ у многочленов $f$ и $g$ следует, что $\exists h(x)=(x-x_0)$, $\phi(x)$, $\psi(x)$, $deg(\phi) = m-1$, $deg(\psi) = n-1$: $g(x)=h(x)\phi(x)$, $f(x)=h(x)\psi(x)$ => $g(x)\psi(x) = f(x)\phi(x)$ $\qed$
\smallskip
(<=): В разложение $\Phi(x) = \phi(x)f(x) = \psi(x)g(x)$ на неприводимые входят все неприводимые множители $f$ и $g$. Но $deg(\Phi) < m + n$, то есть среди этих множителей есть хотя бы один общий. Следовательно, его корень - общий для $f$ и $g$. $\qed$

\bigskip
Рассуждение применимо к кривым $F$ и $G$, если $f$ и $g$ $\forall$ фиксированных $y_0$, $z_0$ имеют те же степени, что и кривые. То есть моном $kx^n$ входит в $f, lx^m$ в $g$ и $f_n*g_m \neq 0$. Применительно к кривым это означает, что ни одна из них не содержит точку (1:0:0). Усилим это условие, потребовав также $(0:0:1) \notin F$, $(0:0:1) \notin G$. Оба условия сохраняют общность, так так система координат выбирается произвольно в пространстве размерности больше двух. Это условие и комментарий отностельно его общности потребуются в дальнейшем, потому заслуживают отдельного обозначения (*) %FIXME - строже
\bigskip

Записав полученное выражение, получаем равенство $(f_n*x^n + ... + f_0)*(\phi_{m-1}*x^{m-1} + ...) = (g_m*x^m + ... + g_0)(\psi_{n-1}*x^{n-1} + ...)$ для произвольных фиксированных $y_0, z_0$
Расписав выражение по степеням $x$, получим систему уравнений на коэффиценты $\phi$, $\psi$.
Теперь можно сформулировать условие наличия у двух кривых точки пересечения.
\smallskip
Обозначим $Syl(F,G)(x,y)$ (матрица Сильвестра (\href{https://ru.wikipedia.org/wiki/%D0%A1%D0%B8%D0%BB%D1%8C%D0%B2%D0%B5%D1%81%D1%82%D1%80,_%D0%94%D0%B6%D0%B5%D0%B9%D0%BC%D1%81_%D0%94%D0%B6%D0%BE%D0%B7%D0%B5%D1%84}{Джозеф Сильвестр})) =

\smallskip
$\begin{pmatrix}
g_m & g_{m-1} & g_{m-2} & \cdots & \cdots & g_0 & 0 & \cdots & 0 \\
0 & g_m & g_{m-1} & g_{m-2} & \cdots & g_1 & g_0 & \cdots & 0 \\
\cdots & \cdots & \cdots & \cdots & \cdots & \cdots & \cdots & \cdots & \cdots \\
0 & \cdots & 0 & g_m & g_{m-1} & g_{m-2} & \cdots & g_1 & g_0 \\
f_n & f_{n-1} & \cdots & \cdots & f_0 & 0 & \cdots & \cdots & 0 \\
0 & f_n & \cdots & \cdots & f_1 & f_0 & 0 & \cdots & 0 \\
\cdots & \cdots & \cdots & \cdots & \cdots & \cdots & \cdots & \cdots & \cdots \\
0 & \cdots & \cdots & \cdots & 0 & f_n & \cdots & f_1 & f_0
\end{pmatrix}$
% Матрица с Википедии.
\bigskip

Пусть $f=f_n*x^n + ... + f_0$, $g=g_m*x^m + ... + g_0$, $f_i \in \mathbb{C}[y,z]$, $g_i \in \mathbb{C}[y,z]$ соответствующей степени - многочлены, определяющие кривые $F$ и $G$,\newline
Тогда $(x_0:y_0:z_0)$ - точка пересечения $F$ и $G$ - существует тогда и только тогда, когда результант = $Res(F,G) := det(Syl(F,G))(y_0,z_0) = 0$. Эта точка однозначно задаётся нулём результанта, так как $\phi(x) и \psi(x)$ по построению зависят от $x_0$, а частное от деления в кольце многочленов определено единственным образом, то есть $h(x)=x-a=f/\psi=g/\phi$.

Заметим, что все рассуждения проходят в однородных координатах и обобщимы на случай произвольной размерности простой серией синтаксических замен.

\textbf{Лемма 2}

\textit{$Res(F,G)$ $\equiv$ $0$ => $F$ и $G$ имеют бесконечное число точек пересечения.}

\textbf{Доказательство леммы 2:}

$Res(F,G) \equiv 0 <=> \forall y,z \in \mathbb{C}$ $\exists x: (x,y,z) \in F \cap G$ - это и означает бесконечное количество точек пересечения. $\qed$
%В таком случае можно говорить, что кривые имеют общую компоненту, так как у F и G как у многочленов от трёх переменных существует общий множитель. %FIXME - раскрыть и обосновать или удалить.

\textbf{Лемма 3}

\textit{$Res(F,G)$ кривых степеней $m$ и $n$ соответственно - однородный многочлен степени $mn$ от $(y,z)$}

\textbf{Доказательство леммы 3:}

Многочлены $f$ и $g$ - однородные.
Тогда $Syl(F,G)(\lambda y,\lambda z)$ =

$\begin{pmatrix}
g_m & \lambda*g_{m-1} & \lambda^{2}*g_{m-2} & \cdots & \cdots & \lambda^{m}g_0 & 0 & \cdots & 0 \\
0 & g_m & \lambda*g_{m-1} & \lambda^{2}*g_{m-2} & \cdots & \lambda^{m-1}*g_1 & \lambda^{m}*g_0 & \cdots & 0 \\
\cdots & \cdots & \cdots & \cdots & \cdots & \cdots & \cdots & \cdots & \cdots \\
0 & \cdots & 0 & g_m & \lambda*g_{m-1} & \lambda^{2}*g_{m-2} & \cdots & \lambda^{m-1}*g_1 & \lambda^{m}*g_0 \\
f_n & \lambda*f_{n-1} & \cdots & \cdots & \lambda^{n}*f_0 & 0 & \cdots & \cdots & 0 \\
0 & f_n & \cdots & \cdots & \lambda^{n-1}*f_1 & \lambda^{n}*f_0 & 0 & \cdots & 0 \\
\cdots & \cdots & \cdots & \cdots & \cdots & \cdots & \cdots & \cdots & \cdots \\
0 & \cdots & \cdots & \cdots & 0 & f_n & \cdots & \lambda^{n-1}*f_1 & \lambda^{n}*f_0
\end{pmatrix}$
\smallskip
Домножим первые $n$ строк на 1, $\lambda$, $\dots$, $\lambda^{n-1}$ соответственно, оставшиеся $m$ строк на 1, $\lambda$, $\dots$, $\lambda^{m-1}$ (учтя соответствующие изменения определителя, который поделится на $\lambda^{1 +\dots + (n-1)}$ и на $\lambda^{1 +\dots + (m-1)}$).
Тем самым столбцы матрицы домножились на 1, $\lambda$, $\dots$, $\lambda^{m + n - 1}$ соответственно.\newline
В итоге $Res(F,G)(\lambda y,\lambda z)$ = $\lambda^{p-q-r}Res(F,G)(y,z)$, где $p - q - r = (m+n)(m+n-1)/2 - m(m-1)/2 - n(n-1)/2 = mn$.
\qed

\bigskip
Сопоставим результанту неоднородный многочлен $R(z)$ той же степени (это возможно из условия *), положив $y=1$.
$R(z)$ - многочлен от одной переменной с комплексными коэффициентами, то есть к нему применима основная теорема алгебры и он имеет $mn$ корней.
$Res(x,y)$ - однородный многочлен, то есть любому его корню $(y_0, z_0)$ соотвествует корень $(1, z_0/y_0) (y_0 \neq 0$ при $z \neq 0$ по условию *)
По построению $R(z_0/y_0) = 0$.
Пусть теперь $\exists x: R(x) = 0$. Тогда точки $(1:x)$ - корни $Res(y,z)$ по тому же построению.
То есть $Res(y,z)$ имеет $mn$ корней.

Отсюда следует утверждение теоремы. $\qed$

Заметим также, что лишь последнее рассуждение перестало быть обобщимым на пространство произвольной размерности.

\subsection{Наглядный смысл} % А нужно ли вообще?
%%\item[Аффинная карта и локальные аффинные координаты]
%%Чтобы сохранить возможность рассуждать о проективной плоскости и кривых на ней образно, введём понятие аффинной карты. \item[Афинной картой] $\mathbb{C}P^2$ называется плоскость $\mathbb{C}^3$, не проходящая через 0 (в аффинном пространстве, связанном с $\mathbb{C}^3$). Точки проективной плоскости изображаются на ней пересечениями соответствующих прямых с ней. Тогда кривая на аффинной карте соответствует некоторой кривой проективной плоскости с точностью до плоскости, параллельной карте и проходящей через 0 - прямые этой плоскости не пересекаются с картой, такая плоскость (а точнее, прямые, проходящие через 0 на ней) называется бесконечно удалённой (проективной) прямой для карты.
%%Аффинная карта порождается пространством решений некоторого неоднородного уравнения v(x) = c. Функционалы на карте, дополняющие v до базиса двойственного к $\mathbb{C}^3$ пространства задают локальные аффинные координаты.

% Блок про аффинные карты закомментировал в самом начале, надеясь, что он мне не понадобится нигде далее. Не хочу смешивать здесь и далее аффинные пространства с векторными, а рисунки, получающиеся на аффинных картах, обобщать сложнее, чем сразу думать в терминах, устойчивых к изменению размерности. В этом разделе, конечно, понадобится.
% Здесь также хорошо будет смотреться описание результанта в аффинных картах. То, что он одноро ден, отражает тот факт, что его решения в количестве mn - точки проективной плоскости, а не исходного пространства. Этот факт надо аккуратнее рассмотреть к моменту написания раздела.
% Описание рисунка
% Некий аналитический этюд про правило Цойтена.
\section{Соотношения Плюккера}
\subsection{Базовые определения и формулировки}
% Определения всех действующих лиц. Начиная с двойственной кривой.
\section{Дальнейшие обобщения}
\end{document}
