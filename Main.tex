\documentclass[a4paper, 12pt]{article}
\usepackage[utf8]{inputenc}
\usepackage[T2A]{fontenc}
\usepackage{amsmath,amssymb,amsthm}
\usepackage[a4paper,hmargin=2.5cm,vmargin=2.5cm]{geometry}
\usepackage[english,russian]{babel}

\begin{document}
%%
%% Title page
%%
\begin{center}
{\scshape Федеральное государственное автономное\\
образовательное учреждение высшего образования\\
<<Национальный исследовательский университет\\
<<Высшая школа экономики>>\\[1ex]
Факультет математики\par}

\par\vfill

\textbf{\large Гузеев Виталий Вячеславович}

\vspace{1.5cm}

{\Large\bfseries
Соотношения Плюккера
\par}

\vspace{1.5cm}

Курсовая работа студента 1 курса\\[1ex]
образовательной программы бакалавриата <<Математика>>
\par\vfill
\noindent\hspace{0.52\textwidth}\parbox[t]{0.48\textwidth}{%
Научный руководитель:\\[3pt]
кандидат физико-математических наук,\\
профессор\\
Городенцев Алексей Львович\\[2ex]
}%
\par\vfill
Москва 2019
\end{center}
\thispagestyle{empty}
\pagebreak
%%
%% ===========================================================================
%%
\section{Теорема Безу}
% Считаю следующую подсекцию примерно завершённой.
\subsection{Базовые определения и формулировки}
\begin{description}
\item[Проективная плоскость]
Рассмотрим векторное пространство $\mathbb{C}^3$. Ненулевые векторы $\mathbb{C}^3$, рассматриваемые с точностью до пропорциональности, образуют двумерное пространство, называемое комплексной (далее подразумевается) проективной плоскостью.
\item[Глобальные однородные координаты]
Координаты коллинеарных векторов в произвольном базисе связаны соотношениями $x_i=\lambda*y_i  \forall i \in {0..2}$. То есть можно говорить о задании точки проективной плоскости соотношением координат векторов порождающего её пространства $(x_0:x_1:x_2)$.
Произвольный многочлен f от трёх переменных над $\mathbb{C}$ не обязан задавать множество точек на проективной плоскости, так как для точки x: f(x) = 0 не гарантируется, что f($\lambda$x) = 0, то есть множество решений может не задавать никаких точек проективной плоскости.
\item[Однородные многочлены] Назовём однородными многочлены, для которых f(x) = 0 => f($\lambda$x) = 0. Альтернативное определение - однородными называются многочлены, все мономы которых имеют одинаковую степень. Эквивалентность определений проверяется напрямую - $\lambda$ в каждый моном войдёт в степени, равной степени монома. Множество нулей однородного многочлена задаёт некоторое множество точек на проективной плоскости.
\item[Плоская (алгебраическая) кривая]
Множество точек плоскости, в частности, проективной, являющихся нулями некоторого многочлена, будем называть плоской кривой. Степенью кривой будем называть степень задающего её многочлена.
%%\item[Аффинная карта и локальные аффинные координаты]
%%Чтобы сохранить возможность рассуждать о проективной плоскости и кривых на ней образно, введём понятие аффинной карты. \item[Афинной картой] $\mathbb{C}P^2$ называется плоскость $\mathbb{C}^3$, не проходящая через 0 (в аффинном пространстве, связанном с $\mathbb{C}^3$). Точки проективной плоскости изображаются на ней пересечениями соответствующих прямых с ней. Тогда кривая на аффинной карте соответствует некоторой кривой проективной плоскости с точностью до плоскости, параллельной карте и проходящей через 0 - прямые этой плоскости не пересекаются с картой, такая плоскость (а точнее, прямые, проходящие через 0 на ней) называется бесконечно удалённой (проективной) прямой для карты.
%%Аффинная карта порождается пространством решений некоторого неоднородного уравнения v(x) = c. Функционалы на карте, дополняющие v до базиса двойственного к $\mathbb{C}^3$ пространства задают локальные аффинные координаты.

% Блок про аффинные карты закомментировал, надеясь, что он мне не понадобится нигде далее. Не хочу смешивать здесь и далее аффинные пространства с векторными, а рисунки, получающиеся на аффинных картах, обобщать сложнее, чем сразу думать в терминах, устойчивых к изменению размерности.
% Вероятно, хочу вынести это рассуждение вместе с правилом Цойтена в наглядный подраздел после доказательства теоремы.
\item[Точка пересечения кривых]
Точкой пересечения n плоских кривых буду называть общий ноль n многочленов, задающих означенные кривые. Кратность пересечения - минимальная из кратностей общего корня этих многочленов.
\end{description}
Введённых определений достаточно, чтобы сформулировать для комплексной проективной плоскости \textbf{теорему Безу}:
\textit{Две плоские кривые степеней m и n, имеющие конечное количество точек пересечения, пересекаются с учётом кратности пересечений ровно в mn точках}
Заметим, что вся построенная терминология прямо переносится на n-мерный случай с заменой понятия "проективная плоскость" на понятие "проективное пространство".
\subsection{Доказательство}
% Хочу уметь на этом месте строить аппарат под доказательство через распад кривой степени n на n прямых. Нашёл, кажется, у Лузгарева.

\subsection{Наглядный смысл}
% Описание рисунка
% Некий аналитический этюд
\section{Соотношения Плюккера}
\subsection{Базовые определения и формулировки}
% Определения всех действующих лиц. Начиная с двойственной кривой.
\section{Дальнейшие обобщения}
\end{document}
