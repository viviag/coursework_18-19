\RequirePackage{cmap}
\documentclass[a4paper, 12pt]{article}
\usepackage[utf8]{inputenc}
\usepackage[T2A]{fontenc}
\usepackage{amsmath,amssymb,amsthm}
\usepackage{comment}
\usepackage[unicode]{hyperref}
\usepackage[a4paper,hmargin=2.5cm,vmargin=2.5cm]{geometry}
\usepackage[english,russian]{babel}
\overfullrule=2cm

\begin{document}
%%
%% Title page
%%
\begin{center}
{\scshape Федеральное государственное автономное\\
образовательное учреждение высшего образования\\
<<Национальный исследовательский университет\\
<<Высшая школа экономики>>\\[1ex]
Факультет математики\par}

\par\vfill

\textbf{\large Гузеев Виталий Вячеславович}

\vspace{1.5cm}

{\Large\bfseries
Соотношения Плюккера для плоских кривых
\par}

\vspace{1.5cm}

Курсовая работа студента 1 курса\\[1ex]
образовательной программы бакалавриата <<Математика>>
\par\vfill
\noindent\hspace{0.52\textwidth}\parbox[t]{0.48\textwidth}{%
Научный руководитель:\\[3pt]
кандидат физико-математических наук,\\
профессор\\
Городенцев Алексей Львович\\[2ex]
}%
\par\vfill
Москва 2019
\end{center}
\thispagestyle{empty}
\pagebreak
%%
%% ===========================================================================
%%
\section{Теорема Безу}
\subsection{Базовые определения и формулировки}
\begin{description}
\item[Проективная плоскость]
Рассмотрим векторное пространство $\mathbb{C}^3$. Ненулевые векторы $\mathbb{C}^3$, рассматриваемые с точностью до пропорциональности, образуют двумерное пространство, называемое комплексной (далее подразумевается) проективной плоскостью и обозначаемое $\mathbb{C}P^2$.
\item[Глобальные однородные координаты]
Координаты коллинеарных векторов в\newline произвольном базисе связаны соотношениями $x_i = \lambda y_i$ $\forall i \in {0..2}$. То есть можно говорить о задании точки проективной плоскости отношением координат векторов $\mathbb{C}$ $(x_0:x_1:x_2)$.
\item[Аффинная карта]
Аффинной картой называется плоскость векторного пространства, с которой могут пересекаться прямые соответствующего проективного пространства. Точка проективного пространства видна в карте как точка плоскости и имеет в этой карте координаты, которые можно получить из однородных, перейдя в векторном пространстве к базису, содержащему ортогональный карте вектор, и зафиксировав соответствующую ему координату отличным от нуля числом. Точки с координатами $(0:y:z)$ не видны в аффинной карте и образуют \textbf{бесконечно удалённую прямую}.
\end{description}
\begin{description}
\item[Однородные многочлены] Однородными называются многочлены, все мономы\newline которых имеют одинаковую степень. Множество нулей однородного многочлена задаёт некоторое множество точек на проективной плоскости.\newline Свойство однородных многочленов --- $f(x)$ = $0$ => $f(\lambda x)$ = $0$.
\item[Плоская (алгебраическая) кривая]
Множество точек проективной плоскости,\newline являющихся нулями некоторого однородного многочлена, будем называть\newline плоской кривой. Степенью кривой будем называть степень задающего её многочлена. Будем говорить также, что многочлен $f$ задаёт кривую $F$, если кривая $F$ представляет собой множество нулей $f$.
\item[Точка пересечения кривых]
Точкой пересечения $n$ плоских кривых будем называть общий ноль $n$ многочленов, задающих означенные кривые.
\end{description}
Введённых определений достаточно, чтобы сформулировать для комплексной проективной плоскости \textbf{слабую теорему Безу}:
\textit{Две плоские кривые степеней $m$ и $n$, имеющие конечное количество точек пересечения, пересекаются не более чем в $mn$ точках}
\pagebreak
\subsection{Доказательство слабой теоремы Безу}
% Хочу уметь на этом месте строить аппарат под доказательство через распад кривой степени n на n прямых. Нашёл у Арнольда здесь (https://books.google.cz/books?id=wrDDBgAAQBAJ&pg=PT87&lpg=PT87&dq=%D1%82%D0%B5%D0%BE%D1%80%D0%B5%D0%BC%D0%B0+%D0%B1%D0%B5%D0%B7%D1%83+%D1%82%D0%BE%D0%BF%D0%BE%D0%BB%D0%BE%D0%B3%D0%B8%D1%87%D0%B5%D1%81%D0%BA%D0%BE%D0%B5+%D0%B4%D0%BE%D0%BA%D0%B0%D0%B7%D0%B0%D1%82%D0%B5%D0%BB%D1%8C%D1%81%D1%82%D0%B2%D0%BE&source=bl&ots=Co0eCf3004&sig=v6v_5eIJMNjOoiE9qcaSPrB4om8&hl=en&sa=X&ved=2ahUKEwikqs6TlcjfAhXCI1AKHet7BawQ6AEwCHoECAMQAQ#v=onepage&q=%D1%82%D0%B5%D0%BE%D1%80%D0%B5%D0%BC%D0%B0%20%D0%B1%D0%B5%D0%B7%D1%83%20%D1%82%D0%BE%D0%BF%D0%BE%D0%BB%D0%BE%D0%B3%D0%B8%D1%87%D0%B5%D1%81%D0%BA%D0%BE%D0%B5%20%D0%B4%D0%BE%D0%BA%D0%B0%D0%B7%D0%B0%D1%82%D0%B5%D0%BB%D1%8C%D1%81%D1%82%D0%B2%D0%BE&f=false).
% Верно ли, что отказ от геометрических построений вовсе приведёт меня к дивизорам на схемах?

Рассмотрим произвольные многочлены с ненулевыми старшими коэффициентами $f=f_nx^n + ... + f_0$, $g=g_mx^m + ... + g_0$ от одной переменной с коэффициентами в $\mathbb{C}$.

\textbf{Лемма 1}

\textit{У $f$ и $g$ есть общий корень тогда и только тогда, когда отображение $h:k[x]_{\leq g-1} \oplus k[x]_{\leq f-1} \rightarrow k[x]_{\leq f+g-1}$, задаваемое формулой $h(\phi,\psi) = \phi f + \psi g$, не сюръективно.}\newline

\textbf{Доказательство леммы:}
\begin{enumerate}
  \item У $f$ и $g$ есть общий корень <=> НОД($f,g$) $\neq$ 1\newline
  (=>): Оба многочлена делятся на $x-x_0$, где $x_0$ --- общий корень. Следовательно, их НОД тоже делится на $x-x_0$ и единицей быть не может. \qed\newline
  (<=): У многочленов нет общих делителей степени больше нуля. Значит, нет общих корней, так как иначе у них был бы общий делитель $x-x_0$. \qed
  \item НОД($f,g$) $\neq$ 1 <=> $h$ не сюръективно.\newline
  (=>): Рассмотрим $h(\phi,\psi)$ от произвольных аргументов. $h(\phi,\psi)=\phi f + \psi g=(x-x_0)(\phi f' + \phi g')$. То есть $deg(h) \geq 1$. \qed\newline
  (<=): Докажем от противного. Пусть теперь НОД($f,g$) = 1. Тогда по теореме о линейном представлении НОДа существуют такие $\phi,\psi$, что $h(\phi,\psi)=1$. Из этого следует сюръективность отображения. \qed
\end{enumerate}
Лемма доказана.

\smallskip
Обозначим матрицу $h$ как $Syl(f,g)$ =

\smallskip
$\begin{pmatrix}
g_m & g_{m-1} & g_{m-2} & \cdots & \cdots & g_0 & 0 & \cdots & 0 \\
0 & g_m & g_{m-1} & g_{m-2} & \cdots & g_1 & g_0 & \cdots & 0 \\
\cdots & \cdots & \cdots & \cdots & \cdots & \cdots & \cdots & \cdots & \cdots \\
0 & \cdots & 0 & g_m & g_{m-1} & g_{m-2} & \cdots & g_1 & g_0 \\
f_n & f_{n-1} & \cdots & \cdots & f_0 & 0 & \cdots & \cdots & 0 \\
0 & f_n & \cdots & \cdots & f_1 & f_0 & 0 & \cdots & 0 \\
\cdots & \cdots & \cdots & \cdots & \cdots & \cdots & \cdots & \cdots & \cdots \\
0 & \cdots & \cdots & \cdots & 0 & f_n & \cdots & f_1 & f_0
\end{pmatrix}$
% Матрица с Википедии.
\newline
Эта матрица называется матрицей Сильвестра (\href{https://ru.wikipedia.org/wiki/%D0%A1%D0%B8%D0%BB%D1%8C%D0%B2%D0%B5%D1%81%D1%82%D1%80,_%D0%94%D0%B6%D0%B5%D0%B9%D0%BC%D1%81_%D0%94%D0%B6%D0%BE%D0%B7%D0%B5%D1%84}{Джозеф Сильвестр}))

\textbf{Результантом} многочленов $f,g$ называется определитель матрицы Сильвестра $Syl(f,g)$.\newline
\newline
Несюрьективность линейного отображения означает, что его образ не порождает всё пространство (в данном случае многочленов степени не выше данной), то есть определитель его матрицы равен нулю. Соответственно в дальнейшем в качестве критерия наличия у двух многочленов общего корня будет использоваться именно утверждение о равенстве результанта нулю.\newline\bigskip

Рассмотрим теперь однородные многочлены $f$ и $g$ от трёх переменных $x,y,z$, задающие некоторые плоские кривые на $\mathbb{C}P^2$.\newline
Будем рассматривать эти многочлены как многочлены от $x$ с коэффициентами --- мономами от $y,z$.\newline
Обозначим функцию от переменных $y,z$, значением которой является\newline $Res(f(y,z),g(y,z))$, как $Res(f,g)$ = $det(Syl(f,g))$\newline

\smallskip
\textbf{Лемма 2}

\textit{Результант однородных многочленов $f$ и $g$ степеней $m$ и $n$ соответственно --- однородный многочлен степени $mn$ от $y,z$.}

\textbf{Доказательство леммы 2:}\newline

$Syl(f,g)(\lambda y,\lambda z)$ =

$\begin{pmatrix}
g_m & \lambda g_{m-1} & \lambda^{2} g_{m-2} & \cdots & \cdots & \lambda^{m}g_0 & 0 & \cdots & 0 \\
0 & g_m & \lambda g_{m-1} & \lambda^{2} g_{m-2} & \cdots & \lambda^{m-1} g_1 & \lambda^{m} g_0 & \cdots & 0 \\
\cdots & \cdots & \cdots & \cdots & \cdots & \cdots & \cdots & \cdots & \cdots \\
0 & \cdots & 0 & g_m & \lambda g_{m-1} & \lambda^{2} g_{m-2} & \cdots & \lambda^{m-1} g_1 & \lambda^{m} g_0 \\
f_n & \lambda f_{n-1} & \cdots & \cdots & \lambda^{n} f_0 & 0 & \cdots & \cdots & 0 \\
0 & f_n & \cdots & \cdots & \lambda^{n-1} f_1 & \lambda^{n} f_0 & 0 & \cdots & 0 \\
\cdots & \cdots & \cdots & \cdots & \cdots & \cdots & \cdots & \cdots & \cdots \\
0 & \cdots & \cdots & \cdots & 0 & f_n & \cdots & \lambda^{n-1} f_1 & \lambda^{n} f_0
\end{pmatrix}$\newline
$\forall$ $y,z$: $Syl(f,g)(y,z)$ = $0$. Это следует из свойства однородных многочленов для многочленов $f$ и $g$.
\newline
Домножим первые $n$ строк на 1, $\lambda$, $\dots$, $\lambda^{n-1}$ соответственно, оставшиеся $m$ строк на 1, $\lambda$, $\dots$, $\lambda^{m-1}$ (учтя соответствующие изменения определителя, который поделится на $\lambda^{1 +\dots + (n-1)}$ и на $\lambda^{1 +\dots + (m-1)}$).
Тем самым столбцы матрицы домножились на 1, $\lambda$, $\dots$, $\lambda^{m + n - 1}$ соответственно.\newline
В итоге $Res(f',g')(\lambda y,\lambda z)$ = $\lambda^{p-q-r}Res(F,G)(y,z)$, где $p = (m+n)(m+n-1)/2$, $q = m(m-1)/2 $, $r = n(n-1)/2$ => $p-q-r$ = $mn$.
\qed

\smallskip
\textbf{Лемма 3}

\textit{Если результант многочленов $f$ и $g$ тождественно равен нулю, заданные ими кривые $F$ и $G$ имеют бесконечное количество точек пересечения.}

\textbf{Доказательство леммы 3:}\newline
Рассмотрим многочлены $f_{y:z}(x)$ и $g_{y:z}(x)$, полученные из $f$ и $g$ фиксацией произвольных значений $y$ и $z$. Результант этих многочленов для любых $y,z$ равен нулю по условию. По лемме 1 из этого следует, что для любых $y,z$ существует $x$ такой, что $(x,y,z)$ - общий корень $f$ и $g$. Пар $y,z$ с точностью до пропорциональности бесконечно много, а каждому общему корню соответствует по определению точка пересечения $F$ и $G$.\qed\newline

\smallskip
\textbf{Лемма 4}

\textit{Существует система координат такая, что каждой паре $(y':z')$ координат точек пересечения $F$ и $G$, имеющих конечное число точек пересечения, соответствует единственная точка пересечения $(x':y':z').$}\newline

\textbf{Доказательство леммы 4:}\newline
Предположим, что для какой-то пары $(y:z)$ существуют две точки пересечения $(x_1:y:z)$ и $(x_2:y:z)$. Геометрически это означает, что эти две точки лежат на одной прямой, параллельной $O_x$. С этого момента можно рассматривать все такие пары $(y:z)$ и любое количество точек пересечения на заданных ими прямых. Точек пересечения конечное число. Рассмотрим для каждой из них модуль угла наклона соединяющей их прямой, если эта прямая не параллельна $O_x$. Это конечное множество. Значит, оно имеет отличный от нуля минимум.\par Если теперь отклонить $O_x$ в любом направлении на меньший угол, $O_x$ перестанет быть параллельна какой-либо прямой, соединяющей точки пересечения. А значит, искомая система координат построена.\qed\newline

\textbf{Слабая теорема Безу}

\textit{Две плоские кривые степеней $m$ и $n$, имеющие конечное количество точек пересечения, пересекаются не более чем в $mn$ точках.}\newline

\textbf{Доказательство слабой теоремы Безу:}\newline
Рассмотрим в координатах, обусловленных леммой 4, многочлены $f$ и $g$, задающие кривые $F(x,y,z) = 0$ и $G(x,y,z) = 0$, результант которых не равен тождественно нулю. Их результант имеет не более $mn$ корней с точностью до пропорциональности, следовательно, различных пар координат $(y:z)$ точек пересечения заданных ими кривых $F$ и $G$ не более $mn$. Для каждой пары $(y:z)$ у $f_{y:z}(x)$ и $g_{y:z}(x)$ есть единственный общий корень, то есть каждой паре соответствует единственная точка пересечения. \qed\newline

\pagebreak
\subsection{Доказательство сильной теоремы Безу}
Для того, чтобы можно было сформулировать сильную теорему Безу, требуется ввести дополнительные определения.\newline
Пусть нас интересует пересечение двух данных кривых $F$ и $G$ в заданной точке $p$. Рассмотрим произвольную точку $u$, не лежащую ни на одной из данных кривых, и прямую $l$, которая пересекается с $F$, $G$ и $pu$ только в точке $u$. Теперь параметризуем кривые в окрестности $p$ точками $l$, введя на $l$ систему координат с нулём в $p$ и проведя прямые $ul_t=ul(t)$. Получим для каждого t наборы точек пересечения $F$ и $G$ c $ul_t$. Обозначим их как $f_i(t), g_j(t)$ (для всевозможных индексов). Эти функции пробегают в точности все точки кривых в окрестности. \newline %Теорема о неявной функции
Назовём \textbf{порядком функции при стремлении параметра к нулю} число $\alpha$ такое, что $f(x)/t^\alpha \rightarrow const, x \rightarrow 0$\newline
\textbf{Локальным индексом пересечения в точке} называется сумма порядков разностей $f_i(t) - g_j(t)$ при стремлении параметра к нулю для всевозможных пар $(i,j)$\newline
\textbf{Индексом пересечения двух кривых} называется сумма локальных индексов пересечения по всем точкам пересечения.
\bigskip
Это определение требует обоснования и проверки на корректность, которые будут даны несколько позже.
\bigskip
\textbf{Лемма 5}

\textit{Результант многочленов $f$, $g$ от одной переменной равен произведению попарных разностей корней $f$ и $g$ с учётом кратностей}\newline

\textbf{Доказательство леммы 5:}\newline
Рассмотрим оба выражения как многочлены от корней $g$, $f$.\newline 
Разделим результант, рассматриваемый как многочлен от корней $f$ с коэффициентами в $\mathbb{C}[g_i]$, на $f_1 - g_1$. Получим выражение $Res(f,g) = (f_1 - g_1)q + r, r \in \mathbb{C}[g_i]$, так как его степень меньше единицы.
$r$ не зависит от корней $f$ и равен нулю в случае $f_1 = g_1$, а значит, тождественно равен нулю.\newline
Отсюда следует делимость в $\mathbb{C}[g_i][f_i]$, а значит, и в $\mathbb{C}[g_i, f_i]$. Оба многочлена имеют степень $mn$, свойство равенства нулю при совпадении значений двух аргументов сохраняется для $q$, следовательно, они равны. А значит, и числа, рассматриваемые в лемме, равны.\qed\newline

Рассмотим теперь многочлены $f_t(s) = F(u + sl(t))$, $g_t(s) = G(u + sl(t))$ и их результант. Корни этих многочленов -- числа $f_i(t), g_j(t)$. Результант раскладывается в произведение попарных разностей корней, перемножив все эти разности, получаем, что сумма их порядков равна порядку результанта при стремлении $t$ к нулю.\newline
Это обосновывает определение локального индекса пересечения через эту сумму, так как это число совпадает c кратностью корня рассматриваемого результанта как функции от $t$ в $t=0$ (следует из определения порядка).

% Лемма 6: Результант функций от трёх переменных раскладывается в произведение квадратичных множителей, соответствующих попарным разностям корней кривых.
% Лемма 7: Локальный индекс пересечения совпадает с порядком результанта F и G при приближении к искомому корню, то есть со степенью множителя, отвечающего этому корню в записи на языке многочленов.
% Остаток: то, что это число целое и совпадает с кратностью множителя общего результанта (разложимость которого теперь тоже надо доказать подобно частному случаю).
% И корректность в смысле независимости от построения, да. Будет следовать оттуда.

Теперь можно сформулировать теорему:

\textbf{Сильная теорема Безу}\newline

\textit{Индекс пересечения кривых $F$ и $G$ равен $deg(F)deg(G)$.}

Эта теорема следует из написанного выше, так как сумма степеней множителей результанта равна его степени, а результант имеет степень $mn$.

\begin{comment}
  1. Разложение результанта на билинейные множители. См. семинарский листок из соответствующей алгебры.
  2. Определение локальной кратности пересечения. ()
  3. Совпадение кратности пересечения вообще.
  4. Теорема о совпадении с кратностями множителей результанта.
  5. Оттуда немедленно сильная форма теоремы.
  6. И некоторым довольно длинным образом, полностью описанным у Уокера, соотношения Плюккера.
  _Только эти способом относительно легко перейти к соотношениям Плюккера, вероятно._

  Или

  1. Ввести инвариант непрерывных деформаций кривых - число пересечений с учётом кратностей. Та же ловкость рук.
  2. Кратность определить как число точек пересечения в случае, когда обе прямые вырождены. (соответственно нужно подробно описать класс деформаций.)
  3. Теорема Безу получается автоматически после того, как для каждой конкретной точки посчитается эта кратность. Странно и непонятно.

  Или

  1. Как у Ландо. Но чтобы удержать уровень строгости, нужно ввести много нового аппарата.

  Или

  1. Спрямить одну из кривых.
  2. Определить "число пересечений с учётом кратностей" так, чтобы оно было инвариантом диффеоморфизма. Проблема в особых точках.
  3. Рассмотреть случай на прямой. При этом вторая кривая получает степень mn через преобразование координат.

  Или

  1. Раздутие результанта в диагонали.

  Или

  1. Раздутие нашей плоскости в точках пересечения кривых. Кратность определяется по числу точек, в которые перешла данная. Это согласуется с произведением числа касательных, но намного строже.
  2. Записать раздутие уравнениями.
  3. Посчитать число всех точек на воткнутых прямых. Инвариантным способом на уравнениях.
\end{comment}

\section{Использованная литература}
\begin{enumerate}
  \item Уокер Р., Алгебраические кривые // Москва, 1952
  \item Ландо С. К., Алгебраические кривые, курс лекций // Москва, 2011
  \item Gorodentsev A. L., Algebraic Geometry Start Up Course // Moscow, 2009
\end{enumerate}

\end{document}
